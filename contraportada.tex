\newgeometry{
    top    = 5mm,  % marxe superior
    left   = 5mm,  % marxe esquerdo
    right  = 5mm,  % marxe dereito
    bottom = 5mm,  % marxe inferior
    nohead = true, % desactivar o encabezado
    nofoot = true, % desactivar o pe de paxina
}
\onecolumn

% FONDO %%%%%%%%%%%%%%%%%%%%%%%%%%%%%%%%%%%%%%%%%%%%%%%%%%%%%%%%%%%%%%%%%%%%%%
\AddToShipoutPictureBG*{
    \put(0,0){
        \parbox[b][\paperheight]{\paperwidth}{
            \vfill
            \centering
            {\transparent{0.1}\includegraphics{./logos/botafumeiro.png}}
            \vfill
        }
    }
}
% No caso de poñelo en todas as páxinas, pode pararse con \ClearShipoutPicture

{% os textos de agradecemento e similares
\centering
    \vspace*{3em}
    {%
        \fontsize{16pt}{0pt}\selectfont%
        \textit{%
            Lorem ipsum dolor sit amet, consectetur adipiscing elit. Integer dictum%
            pulvinar leo at vehicula. Integer feugiat elementum diam, nec tristique sem%
            tincidunt ut. Praesent nec tortor neque. Vestibulum eu scelerisque tellus.}%
        \par%
    }%

    \vspace*{5em}%
    \adforn{35}%
    \vspace*{5em}%

    {%
        \fontsize{16pt}{0pt}\selectfont%
        \textit{%
            Lorem ipsum dolor sit amet, consectetur adipiscing elit. Integer dictum%
            pulvinar leo at vehicula. Integer feugiat elementum diam, nec tristique sem%
            tincidunt ut. Praesent nec tortor neque. Vestibulum eu scelerisque tellus.}%
        \par%
    }%

    \vspace*{5em}%
    \adforn{35}%
    \vspace*{5em}%

    {%
        \fontsize{16pt}{0pt}\selectfont%
        \textit{%
            Lorem ipsum dolor sit amet, consectetur adipiscing elit. Integer dictum%
            pulvinar leo at vehicula. Integer feugiat elementum diam, nec tristique sem%
            tincidunt ut. Praesent nec tortor neque. Vestibulum eu scelerisque tellus.}%
        \par%
    }%
}

\thispagestyle{empty}
\vspace*{3em}

\vfill
\hrulefill

% :FACER: tal vez facer variables cos contidos das URLs do QR e dos contactos..?

% O do QR non sei por qué pero a veces colocase mal. Mirando con
% lua-visual-debug pareceme que crea un numero incrible de boxes que pode que
% toleen a colocacion doutras cousas. Por agora furrula, polos pelos.
\begin{minipage}[c][5cm]{5cm}
        \hypersetup{urlcolor=black}
        \qrset{height = 4cm}%
        % Que quede para a posteridade que rickrolleei a Celia e a Sebas con este QR
        \qrcode{https://youtu.be/dQw4w9WgXcQ}
\end{minipage}
%
\hfill % non se deben poñer liñas en branco arredor deste \hfill
%
\begin{minipage}[c]{0.4\linewidth}
    \centering
    Informacion de contacto\par
    \url{revistafisicaUSC@gmail.com}\par
    \url{https://github.com/DAF-USC/revista}\par
    \vspace*{3em}
\end{minipage}
%
\hfill
%
% Sobre o logo en formato .eps
%
% O orixinal ven de:
% https://nubeusc.sharepoint.com/sites/servizos-oficina-web/Documentos%20compartidos/Forms/AllItems.aspx?csf=1&web=1&FolderCTID=0x012000441AD9196B55D84292BD3BC4FC87F798&id=%2Fsites%2Fservizos%2Doficina%2Dweb%2FDocumentos%20compartidos%2FImaxe%20corporativa%2FLogotipo%20da%20USC&viewid=a5e177b5%2D7018%2D46d6%2D9352%2Dfc57158bf6b7
% (espero que a ligazón dure)
% O arquivo .eps debe pasarse a PDF con 'epstopdf'. Esto debería facerse
% automáticamente, supoñendo que existe o executable. Prefiro facelo así porque
% o que comparten da USC ten ese formato
\begin{minipage}[c][4cm]{5cm}
    \begin{picture}(5cm,4cm)%
        \put(0,0){\hbox{\includegraphics[width=5cm]{./logos/usc-branco-negro.eps}}}
    \end{picture}
\end{minipage}
